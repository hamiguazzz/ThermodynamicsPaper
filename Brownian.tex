\section{一维布朗运动}
%TODO

\section{一维布朗运动的分布函数}
我们接下来尝试讨论一维布朗运动的分布函数,令
$\mathcal{O}[x(t)]=I\left(x_{1}<x(t)<x_{2}\right)$表示单个粒子在$\left(x_{1}, x_{2}\right)$之间存在的时间,如果满足括号内的条件则$I(\cdots)=1$,不满足则为0.
因此可以沿着粒子的轨迹$x(t)$,通过$I(\cdots)$在值0到1之间的变换,相应的得到粒子是否在区域$\left(x_{1}, x_{2}\right)$中。而这个可观测值I的集合平均,
可以表示为:
\begin{equation}
\left\langle I\left(x_{1}<x(t)<x_{2}\right)\right\rangle=\int_{0}^{\infty} I\left(x_{1}<x<x_{2}\right) P_{t}(x) \mathrm{d} x \sim \int_{x_{1}}^{x_{2}} e^{-\beta V(x)} / \mathcal{Z}_{t}\mathrm{d}x
\end{equation}
当系统处于热平衡时,该式即变为粒子出现在区间$\left(x_{1}, x_{2}\right)$ 的概率,粒子在该区域内停留的时间记为滞留时间:$t_{x_{1}<x<x_{2}}$
经过一段长时间的观测,由Boltzmann测度可得:
\begin{equation}
\lim _{t \rightarrow \infty}\frac{t_{x_{1}<x<x_{2}}}{ t}=\frac{\int_{x_{1}}^{x_{2}} e^{-\beta V(x)} \mathrm{d} x}{Z}
\end{equation}
由Boltzmann-Gibbs理论的Birkhoff遍历假设可知,观测得到有限时间平均值为:
\begin{equation}
\frac{t_{x_{1}<x<x_{2}}}{t}=\frac{\int_{0}^{t} I\left(x_{1}<x(t)<x_{2}\right) \mathrm{d} t}{t}
\end{equation}
该式通过对一组路径求平均获得,而每个对应的轨迹都有自己的滞留时间。因此有:\begin{equation}
\left\langle\frac{t_{x_{1}<x<x_{2}}}{t}\right\rangle=\frac{\left\langle\int_{0}^{t} I\left(x_{1}<x(t)<x_{2}\right) \mathrm{d} t\right\rangle}{t}
\end{equation}
在该值的计算中,可以利用公式:
\begin{equation}
\left\langle I\left(x_{1}<x(t)<x_{2}\right)\right\rangle=\int_{x_{1}}^{x_{2}} P_{t}(x) \mathrm{d} x
\end{equation}
在条件:$\mathcal{Z}_{t}=$ $\sqrt{\pi D t}$下可以得到:
\begin{equation}
\left\langle\frac{t_{x_{1}<x<x_{2}}}{t}\right\rangle \sim \frac{1}{t} \int_{0}^{t} \mathrm{~d} t \frac{\int_{x_{1}}^{x_{2}} e^{-\beta V(x)} \mathrm{d} x}{\mathcal{Z}_{t}}=
2 \frac{\int_{x_{1}}^{x_{2}} \exp [-\beta V(x)] \mathrm{d} x}{\mathcal{Z}_{t}}
\end{equation}
2倍的因子是时间积分的结果,由于长时间的限制,下标可以写作0。
现在考虑的情况为一维情况,当x趋于远处,势趋于平坦的势阱。可以界定一个范围$l_{1}$,当$x>l_{1}$时,
将该区域的势能视为0,即代表在$0<x<l_{1}$处,并且有$\exp \left(-x^{2} / 4 D t\right) \sim 1$。
由粒子的扩散性质,在$0\sim l_{1}$处。粒子由势阱中逃逸,向束缚力趋于零的区域扩散。(需要更改)
但是在$x>l_{1}$区域的粒子同时也会向着内部区域扩散,从而重新返回到非零势能区域内。
这说明任何粒子,无论粒子到达了哪个区域当中,都会回归到$x<(l_{1})$区域内。那么以$N\gg1$的粒子实验,经过有限时间后总可以在$(x>l_{1})$以外的区域发现它,也总可以观察到粒子的返回,所以在边界$(x=l_{1})$附近总有一部分不可忽略的粒子存在。
(与本文所讨论的模型相符:经过有限长的时间,由于边界是时间的二分之一次方形式,边界总可以到达足够的长度使得所有粒子回归的概率为1。)(可以更改)
现在考虑一个可观察的时间$\mathcal{O}[x(t)]$,
对于玻尔兹曼因子它是可积分的,那么有:\begin{equation}
\langle\mathcal{O}(x)\rangle=\frac{\int_{0}^{\infty} \mathcal{O}(x) \exp [-\beta V(x)] \mathrm{d} x}{\mathcal{Z}_{t}}
\end{equation}
时间总体平均值为:
\begin{equation}
\langle\overline{\mathcal{O}[x(t)]}\rangle=2\langle\mathcal{O}(x)\rangle
\end{equation}
2倍的因子实际为扩散性质的结果,它直接导致对时间相关配方函数的积分。这其实是伯克霍夫定律的推广(细说一下伯克霍夫定律)。
而这个时间总体平均值等于长时间限制中的整体平均。

之前提到过,时间平均值随着粒子的轨迹变化,讨论在不同轨迹间的波动。
考虑方程
\begin{equation}
\mathcal{O}[x(t)]=I\left(x_{1}<x(t)<x_{2}\right)
\end{equation}
出于简单,令$x_{1}=0$,则$\left(x_{1}, x_{2}\right)$变为了$\left(0, x_{2}\right)$。定义滞留时间$\tau^{\mathrm{in}}$记为\begin{equation}
I\left(x_{1}<x(t)<x_{2}\right)=1
\end{equation}
离散时间$\tau^{\mathrm{out}}$当$I\left(x_{1}<x(t)<x_{2}\right)=0$。
因此在
$\left(0, x_{2}\right)$
区域内粒子处于两种状态的序列为:
\begin{equation}
\left\{\tau_{1}^{\mathrm{in}}, \tau_{1}^{\mathrm{out}}, \tau_{2}^{\mathrm{in}}, \tau_{2}^{\mathrm{out}}, \cdots\right\}
\end{equation}
由于朗之万方程的时间相关性随时间的增长成指数倍形式下降,因此这些时间可被视为互相独立,相同分布的随机变量。
使用$\psi_{\text {out } / \text { in }}(\tau)$来作为$\tau^{\mathrm{out}} 和\tau^{\mathrm{in}}$概率密度函数,在平坦势的情形下$\tau^{\mathrm{out}} $的概率密度函数正比于$t^{-3/2}$($\tau^{out}\gg\tau^{in}$所以$\tau^{out}\sim t$)。
对于固定的测量时间t而言,假设粒子由内部区域到外部区域的次数总共有n 次,那么当t很大时,由于离散时间远大于滞留时间。
因此n的分布由离散时间来确定。
可以重新定义:$I\left(x_{1}<x(t)<x_{2}\right)$的时间平均值记为滞留时间的总和除以测量时间t:
\begin{equation}
\bar{I}=\sum_{i=1}^{n} \tau_{i}^{\mathrm{in}} / t
\end{equation}
注意有效逗留时间:
\begin{equation}
\left\langle\tau_{\mathrm{eff}}\right\rangle=\int_{0}^{t} \tau \psi_{\mathrm{out}}(\tau) \mathrm{d} \tau \propto t^{1 / 2}
\end{equation}
所以有:
\begin{equation}\langle n\rangle \sim t /\left\langle\tau_{\text {eff }}\right\rangle \propto t^{1 / 2}
\end{equation}
再反过来看
\begin{equation}
\langle\bar{I}\rangle \simeq\left\langle\tau^{\mathrm{in}}\right\rangle\langle n\rangle / t
\end{equation}
有时间平均值与$t^{-1/2}$成正比。
现在考虑测量时间t很长时的情况,对于任何j≠i,都有:
\begin{equation}
\left\langle\left(\tau_{i}^{\mathrm{in}}\right)^{2}\right\rangle=\left\langle\left(\tau_{j}^{\mathrm{in}}\right)^{2}\right\rangle
\end{equation}
以及
\begin{equation}
\left\langle\tau_{i}^{\mathrm{in}} \tau_{j}^{\mathrm{in}}\right\rangle=\left\langle\tau_{i}^{\mathrm{in}}\right\rangle^{2}
\end{equation}
可以得到:
\begin{equation}
\left\langle\bar{I}^{2}\right\rangle =\frac{\left\langle\left(\sum_{i=1}^{n} \tau_{i}^{\mathrm{in}}\right)^{2}\right\rangle}{t^{2}}
=\frac{\langle n\rangle\left\langle\left(\tau^{\mathrm{in}}\right)^{2}\right\rangle+\langle n(n-1)\rangle\left\langle\tau^{\mathrm{in}}\right\rangle^{2}}{t^{2}} .
\end{equation}
考虑时间平均值的方差:
\begin{equation}
\left\langle(\bar{I})^{2}\right\rangle-\langle\bar{I}\rangle^{2}=\frac{\left\langle\left(t_{r}\right)^{2}\right\rangle-\left\langle t_{r}\right\rangle^{2}}{t^{2}}=\frac{\left\langle n^{2}\right\rangle-\langle n\rangle^{2}}{t^{2}}\left\langle\tau^{\mathrm{in}}\right\rangle^{2}+\frac{\langle n\rangle}{t^{2}} \underbrace{\left[\left\langle\left(\tau^{\mathrm{in}}\right)^{2}\right\rangle-\left\langle\tau^{\mathrm{in}}\right\rangle^{2}\right]}_{\operatorname{Var}\left(\tau^{\mathrm{im}}\right)}
\end{equation}
由前文得到的n与t的关系,发现第二项相比于第一项可以忽略不计,再由
\begin{equation}
\langle\bar{I}\rangle \simeq\left\langle\tau^{\mathrm{in}}\right\rangle\langle n\rangle / t
\end{equation}
可以得到:
\begin{equation}
\frac{\left\langle\bar{I}^{2}\right\rangle-\langle\bar{I}\rangle^{2}}{\langle\bar{I}\rangle^{2}} \rightarrow \frac{\left\langle n^{2}\right\rangle-\langle n\rangle^{2}}{\langle n\rangle^{2}}
\end{equation}
将其继续到高阶矩,有:
\begin{equation}
\frac{\bar{I}}{\langle\bar{I}\rangle} \rightarrow \frac{n}{\langle n\rangle} \equiv \eta
\end{equation}
这表明在$\left(x_{1}, x_{2}\right)$区域内的滞留时间比上平均滞留时间,实际上等于在其平均值上粒子从滞留态到离散态的数目。可以得到$0<\eta $的概率密度函数\cite{godrecheStatisticsOccupationTime2001},结合$\psi_{\text {out }}(\tau) \propto \tau^{-3 / 2}$
可以得到:\begin{equation}
\operatorname{PDF}(\eta)=\frac{2}{\pi} e^{-\eta^{2} / \pi}
\end{equation}
虽然它有与高斯中心极限定理有关的形式,但是实际上PDF是关于Mittagle-Leffler分布,它与L有关。
Aaronson-Darlin-Kac定理预测在无穷测度的过程的时间平均分布将由Mittag-Leffler分布给出:
\begin{equation}
\mathscr{N}_{\alpha}(\eta)=\frac{\Gamma^{1 / \alpha}(1+\alpha)}{\alpha \eta^{1+1 / \alpha}} l_{\alpha, 0}\left[\frac{\Gamma^{1 / \alpha}(1+\alpha)}{\eta^{1 / \alpha}}\right], l_{\alpha, 0}(\#)
\end{equation}
为利维密度,指数α由$\psi_{\text {out }}(\tau) \propto \tau^{-1-\alpha}$ 给出,此时α=1/2,
那么在我们的物理模型下考虑n个独立的同分布随机变量,根据levy中心极限定理\cite{metzlerRandomWalkGuide2000},相应的概率密度函数Probability Density Function(PDF)为:\begin{equation}
l_{1 / 2,0}(\tau)=\frac{1}{2 \sqrt{\pi}} \tau^{-3 / 2} \exp \left(-\frac{1}{4 \tau}\right)
\end{equation}
现在引入随机变量$y=\sum_{i=1}^{n} \tau_{i} / n^{2}$,由于
\begin{equation}
t=\sum_{i=1}^{n} \tau_{i}
\end{equation}
因此$y=t / n^{2}$,
可以得到PDF关于y的函数形式为:
\begin{equation}
\operatorname{PDF}(n)=\operatorname{PDF}(y)\left|\frac{\mathrm{d} y}{\mathrm{~d} n}\right|=
l_{1 / 2,0}\left(\frac{t}{n^{2}}\right)\left|\frac{2 t}{n^{3}}\right|=\frac{1}{\sqrt{\pi t}} \exp \left(-\frac{n^{2}}{4 t}\right)
\end{equation}