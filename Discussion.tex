\section{讨论}
本文虽然建立了在非禁闭势场下的涨落耗散定理和涨落定理,但由于能力限制,只讨论了一维布朗运动的情况。同时,为保证讨论是有意义的,我们要求平衡态是存在的,这相当于要求势能比1/x衰减的更快,因此对于衰减缓慢的,比如对数分之一的势场,由于其不存在稳定的末态,我们无能为力。

对涨落耗散定理的讨论依赖于势能的泰勒展开,因此如果势能展开近似程度不好,对这种情况需要重新计算式(\ref{equ:ang-x-x2}),没有包括在文章内。在对涨落定理的讨论中,由于计算路径积分的手段较为复杂,我们没有通过数值结果验证一维布朗运动的涨落定理。

此外,本文没有讨论粒子之间的相互作用。除了背景势的形式会产生紧闭问题,相互作用力也有类似的问题。当粒子之间的相互作用是长程力形式时,构造合适的平衡态统计力学和非平衡态统计力学,对这方面的研究可以参考其他文章。\cite{laiThermodynamicEquilibriumLongrange1986,suenThermodynamicEquilibriumLongrange1987,suenThermodynamicEquilibriumLongrange1987a}