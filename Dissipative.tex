\section{涨落耗散定理}
根据以上讨论,我们尝试在一维布朗运动的情况下建立涨落耗散定理。定理的建立主要依赖于坐标空间的位置分布函数。

在长时间演化后,系统可以认为已经达到平衡状态,可以用分布函数描述系统的状态。根据当前问题中分布函数式(\ref{equ:PDF})
的形式,我们仿照此书\cite{zhaoTongJiReWuLiXue2019}中
的方法,求出其在线性响应下的涨落关系。

系统中共轭的外力$f$和位移$x$对能量的影响,在线性响应的情况下可以记为
\begin{equation}
    E(x)=E_0(x)-xf
\end{equation}
因此,在附加较小外力的情况下,分布函数会变为
\begin{equation}
    P(x, t) =\frac{e^{-\frac{x^{2}}{4 D t}-\beta V(x)+xf}}{N(t)}
    \label{equ:pdf-xf}
\end{equation}
其中$\beta=\mathrm{k}_{0} T$。

由于我们研究的系统处于势能函数底部范围附近,我们可以将势能函数在最低点做泰勒展开,
\begin{equation}
    V(x)=V_0+V_1 x+V_2/2 x^2 + \mathcal{O}(x^3)
\end{equation}
代入式(\ref{equ:pdf-xf})并对比标准正态分布公式可以得到在力$f$扰动的存在的情况下,位移$x$发生的响应为
\begin{equation}
    \left\langle x \right\rangle = (\beta f - \beta V_1)\cdot\left( \frac{1}{2Dt} + \beta V_2\right)^{-1}
    \label{equ:ang-x}
\end{equation}

\begin{equation}
    \left\langle x^2 \right\rangle = \left( \frac{1}{2Dt} + \beta V_2\right)^{-1}
\end{equation}

因此,可以得到
\begin{equation}
    \frac{\left\langle x \right\rangle}{\left\langle x^2 \right\rangle} = \beta (f - V_1)
    \label{equ:ang-x-x2}
\end{equation}

可以发现,相对比标准的玻尔兹曼分布,我们会有一个附加项。附加项的存在是可以理解的,考虑在一个力的扰动下,线性响应的位移在势场中产生了非平凡的效果,总的效果相当于抵消了一部分外部力引起的扰动。

在线性响应条件下, 位移$x$对力$f$产生的响应可以表达为
$$
\left\langle x(t) \right\rangle_{f}=\int_{-\infty}^{\infty} \chi\left(t-t^{\prime}\right) f\left(t^{\prime}\right) \mathrm{d} t^{\prime}
$$

其中$\chi(t)=\Theta(t) y(t)$,$\Theta(t)$表示阶跃函数。

由于Kramers–Kronig关系\cite{kramersDiffusionLumierePar1927,kronigTheoryDispersionXrays1926}的结果只依赖于线性响应的假设,因此其仍有
\begin{equation}
    \operatorname{Re} \tilde{\chi}(\omega)=\frac{1}{2 \pi} \int_{-\infty}^{\infty} \mathrm{d} \omega^{\prime} \frac{\tilde{y}_{0}\left(\omega^{\prime}\right)}{\omega^{\prime}-\omega}=\frac{1}{\pi} \int_{-\infty}^{\infty} \mathrm{d} \omega^{\prime} \frac{\operatorname{Im} \tilde{\chi}\left(\omega^{\prime}\right)}{\omega^{\prime}-\omega}
    \label{equ:Kramers–Kronig}
\end{equation}

另外Wiener–Khinchin定理\cite{wienerGeneralizedHarmonicAnalysis1930}不涉及分布函数的具体形式,因此仍有

\begin{equation}
    \left\langle x^{2}\right\rangle=\frac{1}{2 \pi} \int_{-\infty}^{\infty} \mathrm{d} \omega\left\langle|\tilde{x}(\omega)|^{2}\right\rangle=\frac{1}{2 \pi} \int_{-\infty}^{\infty} \mathrm{d} \omega \tilde{C}_{x x}(\omega)
    \label{equ:Wiener–Khinchin}
\end{equation}

假设力$f_t(t) = f\cdot\theta(1-t)$。在线性响应的情况下,$f$产生的响应有$\langle x\rangle_{f}=f \operatorname{Re} \tilde{\chi}(0)$,但同时考虑到$V$引起的对$f$引起变化的减弱,综合考虑式子(\ref{equ:ang-x})可以得到

\begin{equation}
    \langle x\rangle_{f}=(f-V_1) \operatorname{Re} \tilde{\chi}(0)
    \label{equ:ang-x-respone}
\end{equation}

因此,可以综合式(\ref{equ:ang-x-x2},\ref{equ:ang-x-respone},\ref{equ:Kramers–Kronig},\ref{equ:Wiener–Khinchin})得到涨落耗散定理:

\begin{equation}
    \tilde{C}_{x x}(\omega)=2 k_{0} T \frac{\operatorname{Im} \tilde{\chi}(\omega)}{\omega}
\end{equation}

在此系统做如上近似之后,仍然有涨落耗散定理成立。